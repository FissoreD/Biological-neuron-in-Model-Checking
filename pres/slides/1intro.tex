\section{Introduction}

\begin{frame}
  \frametitle{Objective}

  \textit{
    \begin{enumerate}[<+->]
      \item Choisir deux parmi les mini-circuits proposés dans la figure 1 de l'article FCS.pdf, les implémenter en PRISM et tester des propriétés de logique temporelle concernant ces mini-circuits.
      \item Implémenter avec le model checker probabiliste PRISM un neurone biologique de type LI\&F.
    \end{enumerate}
  }

\end{frame}


% \begin{frame}
%   \frametitle{The LI\&F neuron}

%   \begin{definition}[LI\&F: Leaky Integrate and Fire Model]
%     A neuronal network represented by a digraph.
%   \end{definition}

%   \begin{itemize}
%     \item Nodes represent the neurons
%     \item Edges (the synaptic connections) have either positive (activators) or negative (inhibitors) weights
%     \item Nodes contain a membrane potential: if its threshold is overcome, then a spike is emitted
%     \item The leak factor reduces at each time unit the neuron potential
%   \end{itemize}


% \end{frame}

\begin{frame}
  \frametitle{Some existing archetypes}

  \begin{figure}
    \includegraphics<1>[width=0.5\textwidth]{pic/Archetypes1.png}
    \includegraphics<2>[width=0.5\textwidth]{pic/Archetypes2.png}
    \includegraphics<3>[width=0.5\textwidth]{pic/Archetypes3.png}
  \end{figure}

  \footnotetext[1]{Img. from \textit{On the Use of Formal Methods to Model and Verify Neuronal Archetypes}}

\end{frame}